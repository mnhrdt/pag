%\documentclass[a4paper,11pt]{article}
%\usepackage[utf8]{inputenc}
%\usepackage[french]{babel}
%\usepackage{amsmath,xfrac}
%\usepackage[osf,sups]{Baskervaldx} % lining figures
%\usepackage[bigdelims,cmintegrals,vvarbb,baskervaldx,frenchmath]{newtxmath} % math font
%%\usepackage[cal=boondoxo]{mathalfa} % mathcal from STIX, unslanted a bit
%%\usepackage{float}
%\usepackage{graphicx}
%\usepackage{url,hyperref}
%\usepackage{color}
%
%\setlength{\parindent}{0em}
%\setlength{\parskip}{1em}
%%\addtolength{\hoffset}{-3.5em}
%%\addtolength{\textwidth}{7em}
%%\addtolength{\voffset}{-5em}
%%\addtolength{\textheight}{10em}
%
%%\addtolength{\voffset}{-6em}
%%\addtolength{\textheight}{12em}
%
%\begin{document}
%\thispagestyle{empty}
%
%\begin{center}
%	\Large Vibrations d'un système discret
%\end{center}
\title{Vibrations d'un système discret}

On considère les trois matrices suivantes:
\[
	D_3=\begin{pmatrix}
		-2 & 1 & 0 \\
		1 & -2 & 1 \\
		0 & 1 & -2 \\
	\end{pmatrix}
	\qquad
	N_3=\begin{pmatrix}
		-1 & 1 & 0 \\
		1 & -2 & 1 \\
		0 & 1 & -1 \\
	\end{pmatrix}
	\qquad
	P_3=\begin{pmatrix}
		-2 & 1 & 1 \\
		1 & -2 & 1 \\
		1 & 1 & -2 \\
	\end{pmatrix}
\]

{\bf Exercice 1.}
Calculez explicitement (à la main) les valeurs et vecteurs propres des
matrices~$X_3$, pour~$X=D,N,P$.

{\sl Indication: les vecteurs propres sont des évaluations des fonctions
trigonométriques sur des positions régulièrement espacées.}

{\bf Exercice 2.}
Définissez des matrices tridiagonales~$D_n$, $N_n$ et $P_n$, de taille~$n\times
n$ pour~$n\ge 3$, qui généralisent le cas particulier~$n=3$.

{\bf Exercice 3.}
À la vue de l'exercice~$1$, trouvez les valeurs et vecteurs propres des
matrices~$D_n$, ~$N_n$ et~$P_n$ pour~$n\ge 3$.

{\bf Définition.}
Si~$A$ est une matrice positive, on dénote par~$\lambda_k(A)$, $k=1,2,\ldots$
les valeurs propres strictement positives de~$A$, ordonnées par ordre croissant.

{\bf Exercice 4.}
Vérifiez que les matrices~$-D_n$, $-N_n$ et $-P_n$ sont positives et donnez une
expression fermée pour les~\emph{partiels}
\[
	\mu_k(X_n) := \sqrt{\frac{\lambda_k(-X_n)}{\lambda_1(-X_n)}}
\]
pour~$X=D,N,P$ et~$k\ge 1$.  Démontrez que quand~$n>>k$ on a~$\mu_k\approx k$,
et estimez la qualité de cette approximation (selon la valeur de~$\tfrac kn$).

{\bf Exercice 5.}
Donnez une interprétation physique à ces matrices et à leurs vecteurs propres.

{\bf Exercice 6.}
Donnez une version continue de toutes ces constructions.

\clearpage
{\bf\large solutions}


{\color{blue}
{\bf Exercice 1.}
Calculez explicitement (à la main) les valeurs et vecteurs propres des
matrices~$X_3$, pour~$X=D,N,P$.
}

On s'intéresse plutôt aux matrices~$-X_n$ car il se trouve qu'elles sont
positives.

\fbox{$D_3$}

%Calculons d'abord les valeurs propres de~$-D_3$, qui sont les racines du
%polynôme caractéristique:
%\[
%	\det\left(\lambda I-D_3\right)
%	=
%	\begin{pmatrix}
%		\lambda-2 & 1 & 0 \\
%		1 & \lambda-2 & 1 \\
%		0 & 1 & \lambda-2 \\
%	\end{pmatrix}
%	=
%	(\lambda-1)^3-2(\lambda-2)
%	=(\lambda - 2)(\lambda^2-4\lambda+2)
%\]
%et les racines sont donc, par ordre croissant
%\[
%	\lambda_1 = 2-\sqrt2
%	\qquad
%	\lambda_2 = 2
%	\qquad
%	\lambda_3 = 2+\sqrt2
%\]
Les vecteurs propres %~$\varphi_1,\varphi_2,\varphi_3$
sont déterminés à un
facteur multiplicatif près.  Par symétrie (la matrice~$D_3$ est invariante par
conjugation
avec~$P=\left(\begin{smallmatrix}0&0&1\\0&1&0\\1&0&0\end{smallmatrix}\right)$),
les vecteurs propres sont symétriques ou anti-symétriques, donc de la
forme~$u=(1,0,-1)$ ou bien~$v_\alpha=(1,\alpha,1)$.  On vérifie que~$D_3u=-2u$,
donc~$u$ est un vecteur propre de~$-D_3$ de valeur propre~$2$.   D'autre côté,
on a~$D_3
v_\alpha
%\left(\begin{smallmatrix}1\\\alpha\\1\end{smallmatrix}\right)
=
\left(\begin{smallmatrix}\alpha-2\\2-2\alpha\\\alpha-2\end{smallmatrix}\right)
=
(\alpha-2)
v_\frac{2-2\alpha}{\alpha-2}
%\left(\begin{smallmatrix}1\\\frac{2-2\alpha}{\alpha-2}\\1\end{smallmatrix}\right)
$.  Si~$v_\alpha$ est un vecteur propre il faudra donc
que~$\alpha=\frac{2-2\alpha}{\alpha-2}$, ou équivalemment~$\alpha=\pm\sqrt 2$
et les valeurs propres associées sont~$2\pm\sqrt 2$.
Ainsi, les trois valeurs
propres de~$-D_3$ sont, par ordre croissant
\[
	\lambda_1 = 2-\sqrt2
	\qquad
	\lambda_2 = 2
	\qquad
	\lambda_3 = 2+\sqrt2
\]
et les vecteurs propres correspondants, normalisés avec la norme sup:
\[
	\varphi_1=\begin{pmatrix}
		\sfrac{\sqrt2}2\\
		1\\
		\sfrac{\sqrt2}2
	\end{pmatrix}
	\qquad
	\varphi_2=\begin{pmatrix}
		1\\
		0\\
		-1
	\end{pmatrix}
	\qquad
	\varphi_3=\begin{pmatrix}
		\sfrac{\sqrt2}2\\
		-1\\
		\sfrac{\sqrt2}2
	\end{pmatrix}
\]
Observez que le vecteur propre~$\varphi_n$ sont les évaluations
de~$\sin\tfrac{n\theta}2$
sur 4 points equi-espacés~$\theta_k=\tfrac{2\pi k}{4}$; on a
ainsi~$\varphi_n^k=\sin\frac{nk\pi}4,\ k,\!n=1,2,3$.

%SCRIPT cat <<END | gnuplot > f/sind3.pdf
%%SCRIPT set term pngcairo crop size 800,400
%SCRIPT set term pdfcairo crop size 5,2.5
%SCRIPT set samples 1000
%SCRIPT set key bottom left box opaque
%SCRIPT set xtics (0, "{/Symbol p}/2" 0.5*pi, "{/Symbol p}" pi, "3{/Symbol p}/2" 1.5*pi, "2{/Symbol p} "2*pi)
%SCRIPT set ytics (-1, 0, "1/{/Symbol \326}2" sqrt(2)/2, 1)
%SCRIPT set arrow from pi/2, graph 0 to pi/2, graph 1 nohead dt 2
%SCRIPT set arrow from 3*pi/2, graph 0 to 3*pi/2, graph 1 nohead dt 2
%SCRIPT set arrow from pi, graph 0 to pi, graph 1 nohead dt 2
%SCRIPT set arrow from 0, 0 to 2*pi, 0 nohead dt 2
%SCRIPT set arrow from 0, 0 to 2*pi, 0 nohead dt 2
%SCRIPT set arrow from 0, 1/sqrt(2) to 2*pi, 1/sqrt(2) nohead dt 2
%SCRIPT set arrow from 0, -1 to 2*pi, -1 nohead dt 2
%SCRIPT set arrow from 0, 1 to 2*pi, 1 nohead dt 2
%SCRIPT set obj circle at pi/2,1/sqrt(2) radius .06 fc rgb "red" fs solid
%SCRIPT set obj circle at pi,1 radius .06 fc rgb "red" fs solid
%SCRIPT set obj circle at 3*pi/2,1/sqrt(2) radius .06 fc rgb "red" fs solid
%SCRIPT set obj circle at pi/2,1 radius .06 fc rgb "blue" fs solid
%SCRIPT set obj circle at pi,0 radius .06 fc rgb "blue" fs solid
%SCRIPT set obj circle at 3*pi/2,-1 radius .06 fc rgb "blue" fs solid
%SCRIPT set obj circle at pi/2,1/sqrt(2) radius .06 fc rgb "dark-green" fs solid
%SCRIPT set obj circle at pi,-1 radius .06 fc rgb "dark-green" fs solid
%SCRIPT set obj circle at 3*pi/2,1/sqrt(2) radius .06 fc rgb "dark-green" fs solid
%SCRIPT plot [0:2*pi] [-1.2:1.2] sin(1*x/2) lw 3 lc rgb "red", sin(2*x/2) lw 3 lc rgb "blue", sin(3*x/2) lw 3 lc rgb "dark-green"
%SCRIPT END
\begin{center}
\includegraphics[width=.9\textwidth]{f/sind3.pdf}
\end{center}

\clearpage

\fbox{$N_3$}

Les lignes de la matrice~$N_3$ sont de somme nulle, donc~$(1,1,1)$ est un
vecteur propre de valeur propre nulle.  Par le même argument que sur~$D_3$, les
autres vecteurs propres doivent être symétriques ou anti-symétriques.  On voit
d'abord que~$u=(1,0,-1)$ est un vecteur propre de valeur propre~$1$; et que
si~$v_\alpha(1,\alpha,1)$ est un vecteur propre alors~$N_3v_\alpha=
\left(\begin{smallmatrix}\alpha-1\\2-2\alpha\\\alpha-1\end{smallmatrix}\right)=(\alpha-1)v_{\frac{2-2\alpha}{\alpha-1}}$,
donc si~$v_\alpha$ est un vecteur propre
alors~$\alpha=\frac{2-2\alpha}{\alpha-1}$ ou équivalemment~$\alpha=1,-2$.
Pour~$\alpha=1$ on récupère le vecteur propre~$(1,1,1)$ et pour~$\alpha=-2$ on
trouve un vecteur propre~$(1,-2,1)$ de valeur propre~$3$.
Ainsi, les trois valeurs
propres de~$-N_3$ sont, par ordre croissant
\[
	\lambda_1 = 0
	\qquad
	\lambda_2 = 1
	\qquad
	\lambda_3 = 3
\]
et les vecteurs propres correspondants, normalisés ``de façon naturelle'':
\[
	\varphi_1=\begin{pmatrix}
		1\\
		1\\
		1
	\end{pmatrix}
	\qquad
	\varphi_2=\begin{pmatrix}
		\sfrac{\sqrt3}2\\
		0\\
		-\sfrac{\sqrt3}2
	\end{pmatrix}
	\qquad
	\varphi_3=\begin{pmatrix}
		\sfrac12\\
		-1\\
		\sfrac12\\
	\end{pmatrix}
\]
Observez que les composantes du vecteur propre~$\varphi_n$ sont des évaluations
de~$\cos\tfrac{(n-1)\theta}2$ sur 3 points
equi-espacés~$\theta_k=\tfrac{2\pi(2k-1)}6$,
ainsi~$\varphi_n^k=\cos\frac{(n-1)(2k-1)\pi}6,\ k,\!n=1,2,3$.

%SCRIPT cat <<END | gnuplot > f/cosn3.pdf
%SCRIPT set term pdfcairo size 5,2.5
%SCRIPT set samples 1000
%SCRIPT set key bottom left box opaque
%SCRIPT set xtics (0, "{/Symbol p}/3" pi/3, "{/Symbol p}" pi, "5{/Symbol p}/3" 5*pi/3, "2{/Symbol p} "2*pi)
%SCRIPT set ytics (-1, "{/Symbol \326}3/2" -sqrt(3)/2, 0, "1/2" 0.5, "{/Symbol \326}3/2" sqrt(3)/2, 1)
%SCRIPT set arrow from pi/3, graph 0 to pi/3, graph 1 nohead dt 2
%SCRIPT set arrow from 5*pi/3, graph 0 to 5*pi/3, graph 1 nohead dt 2
%SCRIPT set arrow from pi, graph 0 to pi, graph 1 nohead dt 2
%SCRIPT set arrow from 0, 0 to 2*pi, 0 nohead dt 2
%SCRIPT set arrow from 0, 0 to 2*pi, 0 nohead dt 2
%SCRIPT set arrow from 0, 0.5 to 2*pi, 0.5 nohead dt 2
%SCRIPT set arrow from 0, sqrt(3)/2 to 2*pi, sqrt(3)/2 nohead dt 2
%SCRIPT set arrow from 0, -sqrt(3)/2 to 2*pi, -sqrt(3)/2 nohead dt 2
%SCRIPT set arrow from 0, -1 to 2*pi, -1 nohead dt 2
%SCRIPT set obj circle at pi/3,1 radius .06 fc rgb "red" fs solid
%SCRIPT set obj circle at pi,1 radius .06 fc rgb "red" fs solid
%SCRIPT set obj circle at 5*pi/3,1 radius .06 fc rgb "red" fs solid
%SCRIPT set obj circle at pi/3,sqrt(3)/2 radius .06 fc rgb "blue" fs solid
%SCRIPT set obj circle at pi,0 radius .06 fc rgb "blue" fs solid
%SCRIPT set obj circle at 5*pi/3,-sqrt(3)/2 radius .06 fc rgb "blue" fs solid
%SCRIPT set obj circle at pi/3,.5 radius .06 fc rgb "dark-green" fs solid
%SCRIPT set obj circle at pi,-1 radius .06 fc rgb "dark-green" fs solid
%SCRIPT set obj circle at 5*pi/3,.5 radius .06 fc rgb "dark-green" fs solid
%SCRIPT plot [0:2*pi] [-1.2:1.2] cos(0*x/2) lw 3 lc rgb "red", cos(1*x/2) lw 3 lc rgb "blue", cos(2*x/2) lw 3 lc rgb "dark-green"
%SCRIPT END
\begin{center}
\includegraphics[width=.9\textwidth]{f/cosn3.pdf}

\end{center}

\clearpage

\fbox{$P_3$}

Le calcul pour~$-P_3$ est différent puisqu'il y a des valeurs propres répétées:
\[
	\det\left(\lambda I + P_3\right)
	=
	\left|
	\begin{matrix}
		\lambda-2 & 1 & 1 \\
		1 & \lambda-2 & 1 \\
		1 & 1 & \lambda-2 \\
	\end{matrix}
	\right|
	=\lambda(\lambda-3)^2
\]
On a alors un vecteur propre~$(1,1,1)$ de valeur propre~$0$, et un sous-espace
propre de dimension~$2$, associé au valeur propre~$3$.  Par commodité, on peut
choisir une base orthogonale de cet sous-espace constituée par un vecteur
symétrique et un vecteur anti-symétrique.  Ainsi, on obtient les mêmes vecteurs
propres que pour~$N_3$, mais n'oublions pas que c'est un choix essentiellement
arbitraire.  Les valeurs propres sont:
\[
	\lambda_1 = 0
	\qquad
	\lambda_2 = 3
	\qquad
	\lambda_3 = 3
\]
et les vecteurs propres correspondants peuvent être choisis ainsi
\[
	\varphi_1=\begin{pmatrix}
		1\\
		1\\
		1
	\end{pmatrix}
	\qquad
	\varphi_2=\begin{pmatrix}
		\sfrac{\sqrt3}2\\
		0\\
		-\sfrac{\sqrt3}2
	\end{pmatrix}
	\qquad
	\varphi_3=\begin{pmatrix}
		\sfrac12\\
		-1\\
		\sfrac12\\
	\end{pmatrix}
\]
À noter que l'on peut appliquer des rotations d'angle arbitraire autour de
l'axe~$(3,2,3)$ sur les vecteurs~$\varphi_2,\varphi_3$ pour obtenir une famille
de bases orthogonales de vecteurs propres.


%SCRIPT cat <<END | gnuplot > f/sincosp3.pdf
%SCRIPT set term pdfcairo crop size 5,2.5
%SCRIPT set samples 1000
%SCRIPT set key bottom left box opaque
%SCRIPT set xtics (0, "{/Symbol p}/3" pi/3, "{/Symbol p}" pi, "5{/Symbol p}/3" 5*pi/3, "2{/Symbol p} "2*pi)
%SCRIPT set ytics (-1, "{/Symbol \326}3/2" -sqrt(3)/2, 0, "1/2" 0.5, "{/Symbol \326}3/2" sqrt(3)/2, 1)
%SCRIPT set arrow from pi/3, graph 0 to pi/3, graph 1 nohead dt 2
%SCRIPT set arrow from 5*pi/3, graph 0 to 5*pi/3, graph 1 nohead dt 2
%SCRIPT set arrow from pi, graph 0 to pi, graph 1 nohead dt 2
%SCRIPT set arrow from 0, 0 to 2*pi, 0 nohead dt 2
%SCRIPT set arrow from 0, 0 to 2*pi, 0 nohead dt 2
%SCRIPT set arrow from 0, 0.5 to 2*pi, 0.5 nohead dt 2
%SCRIPT set arrow from 0, sqrt(3)/2 to 2*pi, sqrt(3)/2 nohead dt 2
%SCRIPT set arrow from 0, -sqrt(3)/2 to 2*pi, -sqrt(3)/2 nohead dt 2
%SCRIPT set arrow from 0, 1 to 2*pi, 1 nohead dt 2
%SCRIPT set arrow from 0, -1 to 2*pi, -1 nohead dt 2
%SCRIPT set obj circle at pi/3,1 radius .06 fc rgb "red" fs solid
%SCRIPT set obj circle at pi,1 radius .06 fc rgb "red" fs solid
%SCRIPT set obj circle at 5*pi/3,1 radius .06 fc rgb "red" fs solid
%SCRIPT set obj circle at pi/3,sqrt(3)/2 radius .06 fc rgb "blue" fs solid
%SCRIPT set obj circle at pi,0 radius .06 fc rgb "blue" fs solid
%SCRIPT set obj circle at 5*pi/3,-sqrt(3)/2 radius .06 fc rgb "blue" fs solid
%SCRIPT set obj circle at pi/3,.5 radius .06 fc rgb "dark-green" fs solid
%SCRIPT set obj circle at pi,-1 radius .06 fc rgb "dark-green" fs solid
%SCRIPT set obj circle at 5*pi/3,.5 radius .06 fc rgb "dark-green" fs solid
%SCRIPT plot [0:2*pi] [-1.2:1.2] cos(0*x) lw 3 lc rgb "red", sin(1*x) lw 3 lc rgb "blue", cos(1*x) lw 3 lc rgb "dark-green"
%SCRIPT END
\begin{center}
\includegraphics[width=.9\textwidth]{f/sincosp3.pdf}
\end{center}


\clearpage

{\color{blue}
{\bf Exercice 2.}
Définissez des matrices tridiagonales~$D_n$, $N_n$ et $P_n$, de taille~$n\times
n$ pour~$n\ge 3$, qui généralisent le cas particulier~$n=3$.
}

Ce sont des matrices zéro presque partout sauf sur les trois diagonales
centrales (et sur les deux coins pour~$P_n$):

\[
	D_n=\left(\begin{smallmatrix}
		-2 & 1 &   & &  \\
		1 & -2 & 1 & &  \\
		& \ddots & \ddots & \ddots & \\
		& &  & &  & \\
		& & &\ddots &\ddots & 1\\
		& & & &1 &-2 \\
	\end{smallmatrix}\right)
	\quad
	N_n=\left(\begin{smallmatrix}
		-1 & 1 &   & &  \\
		1 & -2 & 1 & &  \\
		& \ddots & \ddots & \ddots & \\
		& &  & &  & \\
		& & &\ddots &\ddots & 1\\
		& & & &1 &-1 \\
	\end{smallmatrix}\right)
	\quad
	P_n=\left(\begin{smallmatrix}
		-2 & 1 &  & & &1  \\
		1 & -2 & 1 & &  \\
		& \ddots & \ddots & \ddots & \\
		& &  & &  & \\
		& & &\ddots &\ddots & 1\\
		1& & & &1 &-2 \\
	\end{smallmatrix}\right)
\]

\clearpage
{\color{blue}
{\bf Exercice 3.}
À la vue de l'exercice~$1$, trouvez les valeurs et vecteurs propres des
matrices~$D_n$, ~$N_n$ et~$P_n$ pour~$n\ge 3$.
}

\fbox{$D_N$}
Par analogie avec~$N=3$, on propose les vecteurs propres~$\varphi_n$
%pour~$n=1,\ldots,N$,
dont les composantes sont définies par
\[
	\varphi_n^k = \sin\frac{kn\pi}{N+1}
	\qquad
	k,n=1,\ldots N
\]
Fixons~$n$.
Il faut vérifier que~$\varphi_n$ est un vecteur propre de~$D_N$ et calculer
sa valeur propre.  C'est un calcul trigonométrique élémentaire~\footnote{Rappelons
	la 
formule~$\sin(a+\epsilon)=\sin a\cos\epsilon+\cos a\sin\epsilon$}.  On sépare le
calcul de~$(D_N\varphi_n)^k$ en
trois cas:~$k=1$, $k=N$ et~$2\le k\le N-1$.


Cas~$2\le k\le N-1$:
\[
	(D_N\varphi_n)^k = \varphi_n^{k+1}+\varphi_n^{k-1}-2\varphi_n^k
	=
	\sin\frac{n(k+1)\pi}{N+1}
	+
	\sin\frac{n(k-1)\pi}{N+1}
	-2\sin\frac{nk\pi}{N+1}
\]
{\small
\[
	=
	\sin\frac{nk\pi}{N\!\!+\!\!1}\cos\frac{n\pi}{N\!\!+\!\!1}
	+
	\cos\frac{nk\pi}{N\!\!+\!\!1}\sin\frac{n\pi}{N\!\!+\!\!1}
	+
	\sin\frac{nk\pi}{N\!\!+\!\!1}\cos\frac{n\pi}{N\!\!+\!\!1}
	-
	\cos\frac{nk\pi}{N\!\!+\!\!1}\sin\frac{n\pi}{N\!\!+\!\!1}
	-2\sin\frac{nk\pi}{N\!\!+\!\!1}
\]
}
\[
	=-\left(2-2\cos\frac{n\pi}{N+1}\right)\sin\frac{nk\pi}{N+1}
\]
\[
	=-\left(2-2\cos\frac{n\pi}{N+1}\right)\varphi_n^k
\]

Cas~$k=1$:
\[
	(D_N\varphi_n)^1 = \varphi_n^2-2\varphi_n^1
	=
	\sin\frac{n2\pi}{N+1}
	-2 \sin\frac{n\pi}{N+1}
%\]
%\[
	=
	2\sin\frac{n\pi}{N+1}\cos\frac{n\pi}{N+1}-2\sin\frac{n\pi}{N+1}
\]
\[
	=-\left(2-2\cos\frac{n\pi}{N+1}\right)\varphi_n^1
\]

Cas~$k=N$ (on utilise~$\sin x=\sin(x-\pi)$):
\[
	(D_N\varphi_n)^N = \varphi_n^{N-1}-2\varphi_n^N
	=
	\sin\frac{n(N-1)\pi}{N+1}
	-2 \sin\frac{nN\pi}{N+1}
\]
\[
	=
	\sin\frac{nN\pi}{N+1}\cos\frac{n\pi}{N+1}
	-\cos\frac{nN\pi}{N+1}\sin\frac{n\pi}{N+1}
	-2\sin\frac{nN\pi}{N+1}
\]
%\[
%	=
%	\sin\frac{nN\pi}{N+1}\cos\frac{n\pi}{N+1}
%	+\cos\frac{n1\pi}{N+1}\sin\frac{n N\pi}{N+1}
%	-2\sin\frac{nN\pi}{N+1}
%\]
\[
	=-\left(2-2\cos\frac{n\pi}{N+1}\right)\varphi_n^N
\]

Et après les trois cas on a~$-D_n\varphi_n=\lambda_n\varphi_n$
avec~$\lambda_n=2-2\cos\frac{n\pi}{N+1}$.


{\color{blue}
{\bf Exercice 4.}
Vérifiez que les matrices~$-D_n$, $-N_n$ et $-P_n$ sont positives et donnez une
expression fermée pour les~\emph{partiels}
\[
	\mu_k(X_n) := \sqrt{\frac{\lambda_k(-X_n)}{\lambda_1(-X_n)}}
\]
pour~$X=D,N,P$ et~$k\ge 1$.  Démontrez que quand~$n>>k$ on a~$\mu_k\approx k$,
et estimez la qualité de cette approximation (selon la valeur de~$\tfrac kn$).
}

Par la question 3, les valeurs propres
sont de la forme~$\lambda_n=2-2\cos\frac{n\pi}{N+1}$.  On a~$\lambda_n>0$
pour~$n>0$ donc~$-D_N$ et~$-P_N$ sont définie positives.  La matrice~$-N_n$ a
un valeur propre 0, et les autres sont positifs, donc elle est définie
positive.  On peu le voir aussi en vérifiant que ces matrices sont
irreductiblement diagonale dominantes.

Avec un développement limité du cosinus~$\cos
x=1-\tfrac12x^2+\tfrac1{24}x^4\cdots$ on a
\[
	\lambda_n
	=
	2-2\cos\frac{n\pi}{N+1}
	=
	\left(\frac{\pi n}{N+1}\right)^2
	-\frac1{12}
	\left(\frac{\pi n}{N+1}\right)^4
	+\cdots
\]
Similairement, avec~$\sqrt{2-2\cos x}=x-\frac1{24}x^3+\cdots$ on a
\[
	\mu_n=\sqrt{\frac{\lambda_n}{\lambda_1}}
	\approx n-\frac{\pi^3}{24}\left(\frac{n}{N+1}\right)^3
\]
donc, pour~$N$ très grand et~$n$ petit par rapport à~$N$, on a~$\mu_n\approx
n$, donc les vibrations sont harmoniques.

{\color{blue}
{\bf Exercice 5.}
Donnez une interprétation physique à ces matrices et à leurs vecteurs propres.
}

Ces matrices modèlent un système de~$N$ masses avec des liens elastiques entre
elles, et la condition de bord donnée.  Les vibrations de ce système
s'expriment comme combinaison lineaire des vibratinons propres, avec chaque
fréquence caractéristique.

{\color{blue}
{\bf Exercice 6.}
Donnez une version continue de toutes ces constructions.
}

Les matrices~$D_i$ sont des discrétisations de la dérivée seconde avec condition
de bord de type Dirichlet:
\[
	(D)
	\begin{cases}
		-u''(x) = \lambda u(x) & x\in]0,2\pi[ \\
		u(x) = 0 & x\in\{0,2\pi\}
	\end{cases}
\]
Les fonctions propres de~$(D)$ sont~$\varphi_n(x)=\sin\frac{nx}2$ pour~$n\ge
1$.  Les valeurs propres correspondants sont~$\lambda_n=n^2/4$.
Les ratios de
chaque partiel avec la fréquence fondamentale
sont~$\mu_n=\sqrt\frac{\lambda_n}{\lambda_1}=n$.

Les matrices~$N_i$ sont des discrétisations de la dérivée seconde avec
condition de bord de type Neumann:
\[
	(N)
	\begin{cases}
		-u''(x) = \lambda u(x) & x\in]0,2\pi[ \\
		u'(x) = 0 & x\in\{0,2\pi\}
	\end{cases}
\]
Les fonctions propres de~$(N)$ sont~$\varphi_n(x)=\cos\frac{nx}2$ pour~$n\ge
0$.  Les valeurs propres correspondants sont~$\lambda_n=n^2/4$.
Les ratios de
chaque partiel avec la fréquence fondamentale
sont~$\mu_n=\sqrt\frac{\lambda_n}{\lambda_1}=n$.

Les matrices~$P_i$ sont des discrétisations de la dérivée seconde sur un
domaine périodique:
\[
	(P)
	\begin{cases}
		-u''(x) = \lambda u(x) & x\in\mathbf{T} \\
	\end{cases}
\]
Les fonctions propres de~$(P)$ sont~$\varphi_n(x)=e^{inx}$
pour~$n\in\mathbf{Z}$.  Les valeurs propres correspondants
sont~$\lambda_n=n^2$.  Ils sont tous de multiplicité~$2$ sauf~$\lambda_0$.  On
peut trouver une base réelle sur l'espace propre engendré par les deux
fonctions~$e^{\pm inx}$: l'ensemble~$\{\sin(nx), \cos(nx)\}$ pour~$n\ge 1$.
Les ratios avec la fréquence fondamentale~$\sqrt{\lambda_1}$ sont tous les
entiers positifs.


\end{document}


% vim:set tw=79 spell spelllang=fr:
