\title{Algèbre Linéaire Numérique}

\vspace{-6em}{
	\tiny
	\hfill
	prière de bien vouloir envoyer des corrections/commentaires à
	\tt enric.meinhardt@ens-paris-saclay.fr
}\vspace{3em}

\newcommand{\ds}{\displaystyle}

\newcommand{\1}{\mathbf{1}}
\newcommand{\N}{\mathbf{N}}
\newcommand{\Z}{\mathbf{Z}}
\newcommand{\Q}{\mathbf{Q}}
\newcommand{\R}{\mathbf{R}}
\newcommand{\C}{\mathbf{C}}
\newcommand{\K}{\mathbf{K}}
\newcommand{\T}{\mathbf{T}}
\newcommand{\M}{\mathbf{M}}
\newcommand{\ud}{\mathrm{d}}
\newcommand{\e}{\mathbf{e}}
\newcommand{\x}{\mathbf{x}}
\newcommand{\y}{\mathbf{y}}
\newcommand{\z}{\mathbf{z}}
\newcommand{\Sp}{\mathrm{Sp}}
%\renewcommand{\Re}{\operatorname{Re}}
%\renewcommand{\Im}{\operatorname{Im}}
\newcommand{\OO}{\mathcal{O}}

\newcommand{\parens}[1]{\left(#1\right)} % (x)
\newcommand{\pairing}[2]{\left\langle #1,#2\right\rangle} % <x,y>

% \abs{x}   ->    |x|
% \Abs{x}   ->   ||x||
% \ABS{x}   ->  |||x|||
\newcommand{\abs}[1]{\left|#1\right|}
\newcommand{\Abs}[1]{\left\|#1\right\|}
\newcommand{\ABS}[1]{{\left\vert\kern-0.25ex\left\vert\kern-0.25ex\left\vert #1 \right\vert\kern-0.25ex\right\vert\kern-0.25ex\right\vert}}



\section{Introduction}

L'objectif de ces notes est d'expliquer les méthodes principales de
\emph{résolution d'un système linéaire} et de~\emph{calcul d'éléments
propres} d'une matrice.  Pour la résolution de systèmes, il y a deux types de
méthodes.  Les~\emph{méthodes directes} construisent le vecteur solution en
faisant un nombre fini d'opérations arithmétiques sur les données d'entrée
qui amènent à la solution exacte.  Les \emph{méthodes itératives}, trouvent
une suite de vecteurs qui converge vers le vecteur solution.  Pour le calcul
des éléments propres, toutes les méthodes en dimension~$\ge 5$ sont
nécessairement itératives.

Afin d'évaluer la performance de ces méthodes, nous introduisons les éléments
de base de l'analyse matricielle.
Nous proposons aussi plusieurs exemples d'application.

{\small\sl Ces notes sont basés partiellement sur un poly antérieur de Lionel
Magnis.}

\subsection{Références}

[AK]
G.~Allaire et S.~M.~Kaber,
\emph{Algèbre linéaire numérique}.
Ellipses 2002.

%	[All]
%		G.~Allaire,
%		\emph{Analyse numérique et optimisation}.
%		Éditions de l'École Polytechnique, 2005.

[Cia]
P.~G.~Ciarlet,
\emph{Introduction à l'analyse numérique matricielle et à
l'optimisation}.
Masson, 1994.

[CH]
R.~Courant und D.~Hilbert,
\emph{Mathematische Physik}.
Springer 1931

[Ser]
D.~Serre,
\emph{Matrices: theory and applications}.
Springer, 2002.\\
Théorie générale des matrices avec un langage moderne.

%[Stra]
%G.~Strang,
%\emph{Introduction to applied mathematics}.
%Wellesley, 1986\\
%Collection énorme et très variée de modèles linéaires.

[W]
H.~Weyl,
\emph{The Classical Groups}.
Princeton University Press, 1939\\
Étude détaillée des sous-groupes du groupe linéaire général.

\clearpage
\subsection{Notations}

\begin{itemize}
	\item $\K$ est le corps~$\R$ ou~$\C$.
	\item $\pairing{\cdot}{\cdot}$ est le produit hermitien
		(si~$\K=\C$) ou scalaire (si~$\K=\R$) sur~$\K^n$.
		%On a donc~$\pairing{\x}{\y}=\sum_i x_i\overline y_i$.
	\item $\M_{p,q}(\K)$ est l'ensemble des matrices sur~$\K$
		avec~$p$ lignes et~$q$ colonnes.
		%Cet ensemble s'identifie de façon
		%naturelle avec les applications linéaires~$\K^p\to\K^q$.
	\item $\M_p(\K)$ est l'ensemble~$\M_{p,p}(\K)$ des matrices carrées.
	\item $A^*$ est la matrice conjuguée de~$A$ (si~$\K=\C$) ou transposée
		(si~$\K=\R$).
		%On a donc~$\parens{A^*}_{ij}=\overline{\parens{A}_{ji}}$.
	\item $GL_n(\K)$ est
	%~\emph{sa majesté le groupe linéaire général},
		l'ensemble des matrices inversibles de taille~$n$.
	\item $O_n(\R)$ est 
		%le~\emph{groupe orthogonal}
		l'ensemble des matrices orthogonales ($A^\top A=I$).
	\item $SO_n(\R)$ est
		%le~\emph{groupe unitaire spécial}
		l'ensemble des matrices orthogonales de déterminant~$1$.
	\item $U_n(\C)$ est
		%le~\emph{groupe unitaire}
		l'ensemble des matrices unitaires ($A^*A=I$).
	\item $SU_n(\C)$ est
		%le~\emph{groupe unitaire spécial}
		l'ensemble des matrices unitaires de déterminant~$1$.
	\item $TS_n(\K)$ (resp.~$TI_n(\K)$) est l'ensemble des matrices
		triangulaires supérieures (resp.~inférieures) de taille~$n$.
	\item $TS^{++}_n(\K)$ (et~$TI^{++}_n(\K)$) est l'ensemble des matrices
		triangulaires dont les éléments diagonaux sont strictement positifs.
	\item $TS^{1}_n(\K)$ (et~$TI^{1}_n(\K)$) est l'ensemble des matrices
		triangulaires dont les éléments diagonaux sont tous~$1$.
	\item $S_n(\R)$ (resp.~$H_n(\C)$) est l'ensemble des matrices
			symétriques~(resp.~hermitiennes), définies par la
			condition~$A^\top=A$ (resp.~$A^*=A$)
	\item $S^{++}_n(\R)$ (resp.~$H^{++}_n(\C)$) est l'ensemble des matrices
			symétriques définies positives~(resp.~hermitiennes définies positives),
		\item $\Sp_\C(A)=\left\{\lambda\in\C\ :\ \exists\x\in\K^n\ :\
			A\x=\lambda\x\right\}$ est le~\emph{spectre} de
			$A\in\M_n(\K)$.
	\item $\Abs{\x}_p$ est la norme~$p$ de~$\x\in\K^n$.
%	\item On dit que une matrice~$A$ est~\emph{normale} si elle commute avec sa
%		conjuguée:~$A^*A=AA*$
\end{itemize}

\paragraph{Interprétation géométrique}

Les matrices de~$\M_{p,q}(\K)$ correspondent aux applications
linéaires~$\K^p\to\K^q$.  Les colonnes d'une matrice~$A\in\M_{p,q}$ sont~$p$
vecteurs de~$\K^q$, image des~$p$ vecteurs de la base canonique de~$\K^p$ par
l'application~$\x\mapsto A\x$.

%Comme cas particulier, les matrices colonne~$\M_{p,1}$ correspondent aux
%vecteurs de~$\K^p$ et les matrices ligne~$\M_{1,q}$ aux formes linéaires
%sur~$\K^q$.  Le produit interne (scalaire ou hermitien) détermine une
%bijection entre l'espace et son dual.
%
%Le produit de matrices correspond à la composition des applications linéaires
%associées.  Ainsi, il est une
%application~$\cdot:\M_{p,q}\times\M_{q,r}\to\M_{p,r}$.  En coordonnées,
%si~$A\in\M_{p,q}$ et~$B\in\M_{q,r}$ alors~$AB=C\in\M_{p,r}$ avec
%\[
%	c_{ij}
%	=
%	\sum_{k=1}^q a_{ik}b_{kj}
%	\qquad
%	\textrm{pour}
%	\qquad
%	\begin{matrix}
%	1\le i\le p\\
%	1\le j\le r\\
%	\end{matrix}
%\]

Si~$A\in\M_p(\K)$ est une matrice carrée, on peut l'interpréter aussi comme
une forme bilinéaire~$\K^p\times\K^p\to\K$ définie par~$(\x,\y)\mapsto
\y^*A\x$.  Le cas particulier~$A=I$ donne le produit scalaire ou hermitien.

Chacun des espaces~$M_{p,q}(\K)$ est un espace vectoriel sur~$\K$ de
dimension~$pq$.  Si~$p=q$ alors le produit de matrices est une opération
interne, compatible avec les opérations d'espace vectoriel et~$M_p(\K)$ a
donc la structure d'algèbre sur le corps~$\K$.  Cette algèbre est toujours
associative mais en général non-commutative.  Si on regarde seulement
l'opération de produit, on trouve des sous-groupes intéressants, dits les
\emph{groupes classiques}.  Ces groupes sont toujours des sous-variétés
de~$\M_n(\K)\approx\K^{n\times n}$ déterminées implicitement par des
équations et inégalités polynomiales.

L'ensemble~$GL_n(\K)$ est \emph{<<sa majesté>> le groupe linéaire
général}~\footnote{[W], page~136}.
Ses éléments sont les applications linéaires inversibles de~$\K^n$.  Elles
satisfont~$\textrm{det}(A)\neq 0$; c'est donc un ouvert de~$\M_n$,
image réciproque de l'ouvert~$\K\setminus\{0\}$ par l'application
continue~$\textrm{det}$.
Les autres groupes classiques sont des sous-groupes de~$GL_n(\K)$.

L'ensemble~$O_n(\R)$ est le~\emph{groupe orthogonal}.
Ses éléments sont les isométries linéaires de~$\R^n$, déterminées par la
condition~$A^\top A=I$.  Les colonnes d'une matrice~$A\in O_n(\R)$ forment
une base orthonormée de~$\R^n$.  La condition~$A^\top A=I$ est un ensemble
de~$n(n+1)/2$ équations (polynômiales de degré 2) sur les coefficients
de~$A$, une pour chaque élément sur ou au dessus de la diagonale de~$I$.
L'ensemble~$O_n(\R)$ est donc une variété compacte de dimension~$n(n-1)/2$.
La condition d'isométrie se traduit par~$\pairing{A\x}{A\y}$ et
par~$\Abs{A\x}=\Abs\x$, relations qui suivent immédiatement de~$A^\top A=I$.
Observons que~$\textrm{det}\parens{O_n(\R)}=\left\{\pm 1\right\}$.
L'ensemble~$O_n(\R)$ a deux composantes connexes.

L'ensemble~$SO_n(\R)$ est le~\emph{groupe spécial orthogonal}.  Ses éléments
sont les isométries linéaires de~$\R^n$ qui conservent l'orientation.  Il est
la composante connexe de~$O_n(\R)$ qui contient l'identité.  Il a donc la
même dimension~$n(n-1)/2$.  En particulier, pour~$n=2$ il est de
dimension~$1$ (les rotations du plan) et pour~$n=3$ il est de dimension~$3$
(les rotations de l'espace).

L'ensemble~$TS_n(\R)\cap GL_n(\C)$ est le~\emph{sous-groupe de Borel
standard}.  Ses élements sont les matrices triangulaires supérieures.  Ce
groupe a~$2^n$ composantes connexes, une pour chaque combinaison de signes
sur la diagonale.  La composante qui contient l'identité est~$TS_n^{++}(\R)$.

L'ensemble~$TS^1_n(\R)$ est aussi un groupe, de dimension~$n(n-1)/2$.
Pour~$n=2$ il est le groupe des cisaillements horizontaux du plan
(\emph{horizontal shears}, en anglais), isomorphe au groupe~$(\R,+)$.
Pour~$n=3$ il est le~\emph{groupe de Heisenberg}.

\begin{exercice}
	L'intersection de sous-groupes est toujours un sous-groupe.  Identifiez
	les sous-groupes de~$GL_n(\R)$ suivants:
	\begin{align*}
		TS_n(\R) \cap O_n(\R) &= \solution{\{\pm I\} \approx \Z/2\Z}\\
		TS_n(\C) \cap U_n(\C) &= \solution{\{e^{i\theta} I\} \approx \R/2\pi\R}\\
		TS_n(\R) \cap SO_n(\R) &= \solution{\{I\}} \\
		TS_n(\C) \cap SU_n(\C) &= \solution{\{I\}} \\
		TS_n(\R) \cap TI_n(\R) &= \solution{\textrm{matrices diagonales
		inversibles} \approx \parens{\R^*}^n}\\
		TS^{++}_n(\R) \cap TI_n(\R) &= \solution{\textrm{diagonales strictement
		positives} \approx \parens{\R^+}^n}\\
		TS^1_n(\R) \cap TI_n(\R) &= \solution{\{I\}}\\
	\end{align*}
\end{exercice}

\begin{exercice}
	Pour chacun des groupes de matrices définis ci-dessus, considérez
	le cas complexe~$\K=\C$, et identifiez les propriétés qui changent par
	rapport au cas réel (notamment, la connexité).
	\solution{En général les groupes de matrices complexes sont ``plus
		connexes'' que les réels.  Ceci est dû au fait que l'ensemble d'unités
	de~$\R$ n'est pas connexe mais celui de~$\C$ l'est.}
\end{exercice}

L'ensemble~$S_n(\R)$ des matrices symétriques n'est pas un groupe: le produit
de deux matrices symétriques n'est pas en général symétrique.  La propriété
de symétrie~$A^\top=A$ se traduit par~$\pairing{A\x}\y=\pairing\x{A\y}$.
Autrement dit, la forme bilinéaire~$(\x,\y)=\y^\top A\x$
associée à une matrice symétrique est symétrique.

\begin{proposition}[Théorème spéctral]
	Toute matrice symétrique réelle~$A\in S_n(\R)$ (resp. hermitienne~$A\in
	H_n(\C)$) a des valeurs propres réels et vecteurs propres orthogonaux.
	C'est à dire, il existe une matrice diagonale~$D$ réelle et une matrice
	orthogonale~$V\in O_n(\R)$ (resp.  unitaire~$V\in U_n(\R)$) telles
	que~$A=V^* D V$.
\end{proposition}

\begin{exercice}
	Démontrez le théorème spectral pour le cas hermitien.\\
	\emph{
	Indications: (1)~démontrez que tout valeur propre doit être réel.
	(2)\!~vérifiez que si~$A$ est hermitienne alors~$A-\lambda I$ l'est aussi.
	(3)\!~vérifiez que si~$A$ est hermitienne
	alors~$\Abs{A\x}_2=\Abs{A^*\x}_2$.
	(4)\!~raisonnez par induction sur~$n$.  Si~$n\ge 1$ alors il
	existent~$\lambda\in\R$ et~$\x\in\C^n$ avec~$\Abs\x_2=1$ tels
	que~$A\x=\lambda\x$.  Trouvez une matrice unitaire~$V$ telle
	que~$V\e^1=\x$, puis vérifiez que~$A_1=V^*AV$ est hermitienne et diagonale
	par blocs hermitiens de taille~$1$ et~$n-1$.
	(5)\!\!~Conclure.
}
\end{exercice}

La démonstration du théorème spectral utilise seulement la
propriété~$A^*A=AA^*$ des matrices hermitiennes.  Le théorème spectral reste
vrai, avec la même démonstration, pour toutes les matrices qui ont cette
propriété fondamentale:

\begin{definition}[matrices normales]
	Une matrice~$A$ est~\emph{normale} si~$A^*A=AA^*$.
\end{definition}

Exemples de matrices normales: les matrices symétriques,
hermitiennes, orthogonales, unitaires, anti-symétriques, anti-hermitiennes.

%Une conséquence fondamentale du théorème spectral est que les matrices
%symétriques définies positives admettent une ``racine carrée'': Si~$A\in
%S_n(\R)$ a tous les valeurs propres positifs alors il existe une
%matrice~$B\in TS_n^{++}(\R)$ telle que~$A=B^\top B$.  On verra beaucoup plus



%\subsection{Les 4 problèmes de l'algèbre linéaire}
%
%Les quatre problèmes fondamentaux de l'algèbre linéaire sont les suivants

\subsection{Complexité}

% complexité = nombre de multiplications

% complexité du produit naive de deux matrices quelconque = pqr
% complexité du produit naive de deux matrices carrées = n^3
% une multiplication en C vaut 4 multiplications en R
% résolution d'un système linéaire triangularire n^2/2 multiplications
% calcul naive d'un determinant n!
% calcul d'un determinant par élimination gaussienne n^2/2
% inversion d'une matrice par la règle de Cramér n!
% inversion d'une matrice par élimination Gaussienne n^3

% note: ``those who know how to multiply know how to invert''

% note: formule de Strassen

% note: recherche actuelle en multiplication de matrices (cf. produit Winograd)

% note: en réalité les multiplications sont plus rapides (!) que les additions

% note: en réalité ce qui importe aujourd'hui n'est pas autant le nombre
% total de calculs mais surtout la localité de l'accès à mémoire/possibilité
% de paraléliser

% note: matrices creuses

\subsection{Étude du spectre}

% définition de spectre, exemples
On suppose connues la définition de valeurs et vecteurs propres d'une matrice
carrée.

% matrices et polynômes

% caractérisation variationnelle du spectre

% domaine de gerschgorin

% théorème de Gauss-Lucas


\subsection{Normes matricielles}

% def. norme matricielle

\begin{definition}[norme]
	Soit~$V$ un espace vectoriel sur~$\K$.
	Une~\emph{norme} sur~$V$ est une application~$p:V\to\R$
	qui satisfait les propriétés suivantes
	\begin{enumerate}
		\item[(i)] $p(\x) = 0\ \implies\ \x=0$
		\item[(ii)] $p(\x)\ge 0$
		\item[(iii)] $p(\x+\y)\le p(\x)+p(\y)$
		\item[(iv)] $p\parens{\lambda\x}=\abs{\lambda}p(\x)$
	\end{enumerate}
	Si toutes les propriétés sont satisfaites sauf (i) on dit que~$p$
	est une~\emph{seminorme}.
\end{definition}

\begin{definition}[norme matricielle]
	Une~\emph{norme matricielle} sur les matrices carrées de dimension~$n$ est
	une application~$p:\M_n(\K)\to\R$ telle que
	\begin{enumerate}
		\item[(i)] $p$ est une norme sur l'espace vectoriel~$\M_n(\K)$
		\item[(ii)] $p(AB) \le p(A) p(B)$
	\end{enumerate}
\end{definition}

L'espace vectoriel~$\M_n(\K)$ est de dimension finie donc toutes les normes
sont équivalentes.  Ceci est très pratique : pour démontrer la convergence
d'une suite de matrices, il suffit de le faire avec la norme pour laquelle la
démonstration est plus facile.  Voyons quelques exemples de normes
matricielles.

% def. norme induite / norme d'opérateur , lemme de composition

\begin{definition}[norme induite]
	Soit~$\Abs\cdot$ une norme en~$\K^n$.  La~\emph{norme induite}
	par~$\Abs\cdot$ sur~$\M_n(\K)$ est la norme de l'opérateur
	linéaire %~$\K^n\to\K^n$
	associé à chaque matrice, notée~$\ABS\cdot$.  Ainsi
	\[
		\ABS{A}
		=
		\sup_{\x\neq0}\frac{\Abs{A\x}}{\Abs\x}
		=
		\max_{\Abs\x=1}\Abs{A\x}
	\]
	La norme induite par la norme~$\Abs\cdot_p$ de~$\K^n$
	s'écrit~$\ABS\cdot_p$.
\end{definition}

\begin{exercice}
	Démontrez que toute norme induite est une norme matricielle, et qu'elle
	satisfait les deux propriétés suivantes
	\begin{enumerate}
		\item[(i)] $\Abs{A\x}\le\ABS A\cdot\Abs\x$
		\item[(ii)] $\ABS I = 1$
	\end{enumerate}
\end{exercice}

\begin{exercice}
	Démontrez que la~\emph{norme de Frobenius} définie par
	\[
		\ABS A_F
		=
		\sqrt{\mathrm{tr}\parens{A^*A}}
		=
		\sqrt{\sum_{i,j}\Abs{a_{ij}}^2}
	\]
	est une norme matricielle mais n'est pas une norme induite.
\end{exercice}

% normes induites par la norme p=1,2,\infty

\begin{exercice}
	Vérifiez les caractérisations suivantes des normes induites classiques:
	\[
		\ABS A_1      = \max_j\sum_i\abs{a_{ij}}
		\qquad
		\ABS A_\infty = \max_i\sum_j\abs{a_{ij}} = \ABS{A^*}_1
		\qquad
		\ABS A_2      = \sqrt{\rho\parens{A^*A}} = \ABS{A^*}_2
	\]
\end{exercice}

% lien entre normes induites et rayon spectral

\begin{exercice}[Lien entre normes induites et rayon spectral]
	Démontrez que %les affirmations suivantes:
	\begin{enumerate}
		\item Pour toute matrice~$A\in\M_n(\K)$ on
			a~$\rho\parens{A^m}=\rho\parens{A}^m$ et~$\rho\parens{\lambda
			A}=\abs\lambda\rho\parens{A}$
		\item Si~$A$ est diagonale alors~$\ABS A_p=\rho(A)$
			pour~$1\le p\le\infty$.
		\item Si~$A$ est normale alors~$\ABS A_2=\rho(A)$.
		\item Si~$n>1$ alors le rayon spectral~$\rho$ n'est pas une norme
			sur~$\M_n(\K)$.
		\item Si~$\ABS\cdot$ est une norme induite sur~$\M_n(\C)$ alors
			\begin{equation}\label{eq:rholessthanABS}
				\rho(A)\le\ABS A
			\end{equation}
		\item${}^*$ L'équation~(\ref{eq:rholessthanABS}) est encore vraie pour
			une norme matricielle quelconque sur~$\M_n(\C)$, ou même
			sur~$\M_n(\R)$.
	\end{enumerate}
\end{exercice}

% théorème de Householder

L'inégalité~\ref{eq:rholessthanABS} dit que~$\rho(A)$ minore
l'ensemble~$\left\{\ABS A\ :\ \ABS\cdot\ \textrm{norme induite}\right\}$.  Le
théorème de Householder montre que c'est en fait la borne inférieure

\begin{theorem}[Householder]
	Pour toute matrice~$A\in\M_n(\C)$ et tout~$\epsilon>0$ il existe une norme
	induite~$\ABS\cdot_{A,\epsilon}$ telle que
	\begin{equation}\label{eq:rholessthanABS}
		\ABS A_{A,\epsilon}\le\rho(A)+\epsilon
	\end{equation}
\end{theorem}

\begin{exercice}[Démonstration du théorème de Householder]$ $
	\begin{enumerate}
		\item Soit~$\Abs\cdot_\alpha$ une norme en~$\K^n$ et~$P\in GL_n(\K)$.
			Démontrez que~$N(\x)=\Abs{P\x}_\alpha$ est une norme en~$\K^n$ et que
			sa norme induite est~$N(A)=\ABS{PAP^{-1}}_\alpha$.
		\item Soit~$A\in\M_n(\C)$ et~$\mu>0$, démontrez qu'il existe~$Q\in
			GL_n(\C)$ et~$D$ diagonale telles que~$Q^{-1}AQ=D+\OO(\mu)$
		\item Démontrez le théorème de Householder.
	\end{enumerate}
\end{exercice}


% convergence de suites de matrices
%% \sum A^n  est convergente (vers 0) sii rho(A) < 1 et alors I-A est inversible
%% rho(A) = \lim |||A^n|||^(1/n)
%% exponentielle d'une matrice

\begin{exercice}[Convergence de suites de matrices]
	Démontrez les propositions suivantes:
	\begin{enumerate}
		\item Pour une norme induite, la condition~$\ABS A<1$ implique que~$I-A$
			est inversible, et l'inverse est la somme de la série
			normalement convergente~$\ds\sum_{k\ge0}A^k$.
		\item Pour toute matrice~$A\in\M_n(\K)$, la
			série~$\ds\sum_{k\ge0}\tfrac1{k!}A^k$ est normalement convergente.
			Sa limite~$e^A$ est une matrice inversible.
		\item $*$ Pour toute norme induite, on a
			toujours~$\ds\rho(A)=\lim_{k\to\infty}\ABS{A^k}^{\frac1k}$
	\end{enumerate}
\end{exercice}


\subsection{Conditionnement de matrices}

Beaucoup de problèmes de modélisation se réduisent à la solution de
l'équation~$Ax=b$.  L'opérateur~$A$ représente un système physique agissant
sur un état~$x$ que l'on veut connaître, et~$b$ est l'observation faite.
Dû à des erreurs de mesure, le vecteur~$b$ est souvent perturbé par le bruit,
ce qui donnera une valeur~$x$ inexacte.

{\bf Exemple.} (issu de [AK], section 5.3)
\begin{displaymath}
A = \left(\begin{array}{cccc}
8 & 6 & 4 & 1 \\
1 & 4 & 5 & 1 \\
8 & 4 & 1 & 1 \\
1 & 4 & 3 & 6
\end{array}\right), \quad b =  \begin{pmatrix}8 \\ 10 \\ 2 \\ 6\end{pmatrix} \quad
\Rightarrow \quad
x = \begin{pmatrix}0 \\ 0 \\ 2 \\ 0\end{pmatrix}
\end{displaymath}
En modifiant légèrement~$b$, on obtient une solution très différente
\begin{displaymath}
A = \left(\begin{array}{cccc}
8 & 6 & 4 & 1 \\
1 & 4 & 5 & 1 \\
8 & 4 & 1 & 1 \\
1 & 4 & 3 & 6
\end{array}\right), \quad b = \begin{pmatrix}8.05 \\ 10 \\ 2 \\ 6\end{pmatrix} \quad
\Rightarrow \quad x = \begin{pmatrix}2 \\ -5.225 \\ 5.5 \\ 1.4\end{pmatrix}
\end{displaymath}

Le~\emph{conditionnement} d'une matrice sert à étudier cet effet.

\begin{definition}[conditionnement d'une matrice]
	Soit~$A$ une matrice inversible et~$\ABS\cdot_\alpha$ une norme induite.  Le
	\emph{conditionnement} de~$A$ relatif à cette norme est
	\[
		\mathrm{cond}_\alpha(A) = \ABS A\ABS{A^{-1}}
	\]
\end{definition}

\begin{proposition}[conditionnement euclidien] Pour toute matrice
	inversible~$A$:
\begin{displaymath}
\mathrm{cond}_2(A) = \sqrt{\frac{\lambda_{\max}(A^*A)}{\lambda_{\min}(A^*A)}}
\end{displaymath}
Si $A$ est normale
\begin{displaymath}
\mathrm{cond}_2(A) = \frac{\abs{\lambda_{\max}(A)}}{\abs{\lambda_{\min}(A)}}
\end{displaymath}
Si $O \in U_n(\K)$ alors
\begin{displaymath}
\mathrm{cond}_2(O) = 1, \quad \mathrm{cond}_2(AO) = \mathrm{cond}_2(OA) = \mathrm{cond}_2(A)
\end{displaymath}
\end{proposition}

%Les deux exercices suivants montrent que, lorsqu'on résout un problème de la
%forme~$Ax=b$, le conditionnement de~$A$ sert à majorer l'erreur de la
%solution à partir du bruit sur~$b$ ou même sur~$A$:
\clearpage
Lorsqu'on résout un problème de la forme~$Ax=b$, le conditionnement de~$A$
sert à majorer l'erreur relatif de la solution~$x$.  On présente deux
versions de cette majoration, selon si on a des petites erreurs d'observation
(sur~$b$) ou de modélisation (sur~$A$):

\begin{exercice}[effet des erreurs d'observation]
	Soit~$\delta b$ une perturbation du second membre et~$\delta x$ l'erreur
	induit sur la solution, de sorte que~$A(x+\delta x)=b+\delta b$.  Démontrez
	que l'erreur relatif augmente d'un facteur au plus~$\mathrm{cond}(A)$:
	\[
		\frac{\Abs{\delta x}}{\Abs{x}}
		\le
		\mathrm{cond}(A)
		\frac{\Abs{\delta b}}{\Abs{b}}
	\]
	Démontrez que cette inégalité est optimale: il existe~$b$ et~$\delta
	b$ tels que l'égalité a lieu.
\end{exercice}


\begin{exercice}[effet des erreurs de modélisation]
	Soit~$\delta A$ une perturbation de la matrice et~$\delta x$ l'erreur
	induit sur la solution, de sorte que~$(A+\delta A)(x+\delta x)=b$.  Démontrez
	que l'erreur relatif augmente d'un facteur au plus~$\mathrm{cond}(A)$:
	\[
		\frac{\Abs{\delta x}}{\Abs{x+\delta x}}
		\le
		\mathrm{cond}(A)
		\frac{\ABS{\delta A}}{\ABS{A}}
	\]
	Démontrez que cette inégalité est optimale: il existe~$b$ et~$\delta
	A$ tels que l'égalité a lieu.
\end{exercice}

\begin{remark}
	D'après ces exercices, il est préférable que le conditionnement de $A$ soit
	proche de 1. Ceci est d'autant plus vrai pour les méthodes itératives où
	les erreurs sont propagées et amplifiées à chaque itération. La technique
	du \emph{préconditionnement} consiste à considérer un système linéaire
	équivalent
	\[
		PAx = Pb
	\]
	où la matrice~$P$, dite le~\emph{préconditionneur} a été choisie de façon
	que $\mathrm{cond} (PA) < \mathrm{cond}(A)$. Il existe plusieurs méthodes
	pour bien choisir $P$, par exemple si~$A$ est à diagonale dominante on peut
	choisir~$P=\mathrm{diag}(A)^{-1}$.
	On pourra consulter à ce sujet [AK, Chapitre 5].
\end{remark}

\begin{remark}
\label{rm:cond}
Les inégalités ci-dessus sont certes optimales, mais elles sont en général
très pessimistes (on dit aussi \emph{conservatives}). Ainsi dans
l'Exemple on a
\begin{displaymath}
\frac{\Abs{\delta x}_{2}}{\Abs{x}{2}} \simeq 964 \frac{\Abs{\delta
b}_{2}}{\Abs{b}_{2}}
\end{displaymath}
ce qui fait déjà beaucoup, mais $\mathrm{cond}_2(A) \simeq 3199$.
\end{remark}

%\begin{exercice}
%On va analyser la matrice $A$ du Laplacien discrétisé à la lumière de cette
%remarque.
%\begin{enumerate}
%\item Montrer que 
%\begin{displaymath}
%	\mathrm{cond}_2(A) \underset{n \rightarrow \infty}{\sim} \frac{4n^2}{\pi^2}
%\end{displaymath}
%de sorte que, quand $n$ est grand, la matrice est très mal conditionnée.
%\item En notant $x = U_{\textrm{approx}}$, montrer que
%\begin{displaymath}
%\lim_{n \rightarrow \infty } \frac{1}{n} \n{\xx}{2}^2 = \int_0^1 u^2(x)dx, \quad \lim_{n \rightarrow \infty } \frac{1}{n} \n{b}{2}^2 = \int_0^1 f^2(x)dx
%\end{displaymath}
%\item Montrer enfin qu'il existe une constante $c > 0$ indépendante de $n$ telle que
%\begin{displaymath}
%\frac{\n{\delta x}{2}}{\n{\xx}{2}} \leq c \frac{\n{\delta b}{2}}{\n{b}{2}}
%\end{displaymath}
%\end{enumerate}
%\end{exercice}


% exemple de mauvais conditionnement

% exemple 2 (matrice de Hilbert)

% déf. nombre de condition (associé à une norme induite)

% prop. exemples de cond2 à partir des rayons spectraux

% perturbation de x selon la perturbation de b (lors de la solution de Ax=b)

% perturbation de x selon la perturbation de A (lors de la solution de Ax=b)

% observation que les inégalités sont ``pessimistes''

% exemple de la dérivée seconde discrète


%\subsection{Exemples de problèmes linéaires}
%
%\subsection{Analyse matricielle}
%
%- groupes de matrices
%
%- théorème spéctral
%
%- rayon spéctral
%
%- normes matricielles
%
%- normes induites
%
%- théorème de Householder
%
%- conditionnement d'une matrice: définition, propriétés, exemple du laplacien
%à 5 points


\clearpage
\section{Factorisation de matrices}

\subsection{Systèmes triangulaires et élimination gaussienne}

\subsection{Factorisation LU}

\subsection{Factorisation de Cholesky}

\subsection{Factorisation QR}

%\section{Décomposition de matrices}

\subsection{Diagonalisation}

\subsection{Décomposition polaire}

\subsection{Décomposition SVD}

\subsection{Décomposition SVD incomplète}


\section{Méthodes itératives}

\subsection{Méthodes itératives de résolution: formulation générale}

\subsection{Méthodes de Jacobi et Gauss-Seidel}

\subsection{Descente de gradient à pas fixe}

\subsection{Descente de gradient à pas optimal}

\subsection{Descente de gradient stochastique}

\subsection{Méthode du gradient conjugué}

\subsection{Méthode de la puissance, puissance inverse}

\subsection{Méthodes de calcul de la SVD}

- bidiagonalisation + méthode itérative

%\subsection{Méthode QR}



\section{Éléments propres et singuliers}

\subsection{Exemples de problèmes de valeurs propres}

- polynômes (et matrice compagnon)

- acp

- analyse quantitative d'une EDO

- oscillations

- spectre d'un corps

\subsection{Valeurs singuliers}

- caractérisation variationnelle des valeurs propres et valeurs singuliers

\subsection{Analyse du spectre}


%\begin{verbatim}
%S4. Algèbre linéaire numérique : généralités
%4.0. Les 4 problèmes de l'algèbre linéaire selon Strang (Ax=b avec
%m=n, m<n, m>n, et Ax=lambda * x)
%4.1. Exemples de problèmes linéaires: minimisation quadratique,
%regression linéaire, EDO et EDP discrètes, splines
%4.2. Analyse matricielle: rayon spéctral, normes, normes induites,
%théorème de Householder
%
%S5. Algèbre linéaire numérique : méthodes de résolution
%5.1. Systèmes triangulaires, Algorithme d'élimination gaussienne
%5.2. Factorisation LU
%5.3. Factorisation Cholesky
%5.4. Méthodes iteratives : formulation générale
%5.5. Jacobi, Gauss-Seidel
%5.6. Gradient, gradient à pas optimal
%5.7. Conditionnement d'une matrice : définition, propriét'es, exemple
%pour le laplacien à 5 points
%
%S6. Algèbre linéaire numérique : éléments propres
%6.1. Exemples de problèmes de valeurs propres : ACP, analyse
%quantitative d'une EDO, oscillations, polynômes
%6.2. Analyse du spectre: cercles de Gerschgörin, Gauss-Lucas
%6.3. Méthode de la puissance, puissance inverse
%6.4. Décomposition QR : existence, unicité, algorithmes
%
%S7. Résolution d'équations non-linéaires
%7.1. Méthodes en dimension 1: bisection, Newton, secante
%7.2. Racines de polynomes d'une variable
%7.3. Cas vectoriel: optimisation vs résolution d'une équation (
%min{F(x)} vs F'(x)=0 ), rappel des multiplicateurs de Lagrange
%7.4. Méthode de Newton et variantes
%\end{verbatim}

% vim:set tw=77 filetype=tex spell spelllang=fr ts=2 sw=2:
